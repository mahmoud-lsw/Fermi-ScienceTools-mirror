\section{\module{functools} ---
         Higher order functions and operations on callable objects.}

\declaremodule{standard}{functools}		% standard library, in Python

\moduleauthor{Peter Harris}{scav@blueyonder.co.uk}
\moduleauthor{Raymond Hettinger}{python@rcn.com}
\moduleauthor{Nick Coghlan}{ncoghlan@gmail.com}
\sectionauthor{Peter Harris}{scav@blueyonder.co.uk}

\modulesynopsis{Higher-order functions and operations on callable objects.}

\versionadded{2.5}

The \module{functools} module is for higher-order functions: functions
that act on or return other functions. In general, any callable object can
be treated as a function for the purposes of this module.


The \module{functools} module defines the following function:

\begin{funcdesc}{partial}{func\optional{,*args}\optional{, **keywords}}
Return a new \class{partial} object which when called will behave like
\var{func} called with the positional arguments \var{args} and keyword
arguments \var{keywords}. If more arguments are supplied to the call, they
are appended to \var{args}. If additional keyword arguments are supplied,
they extend and override \var{keywords}. Roughly equivalent to:
  \begin{verbatim}
        def partial(func, *args, **keywords):
            def newfunc(*fargs, **fkeywords):
                newkeywords = keywords.copy()
                newkeywords.update(fkeywords)
                return func(*(args + fargs), **newkeywords)
            newfunc.func = func
            newfunc.args = args
            newfunc.keywords = keywords
            return newfunc
  \end{verbatim}

The \function{partial} is used for partial function application which
``freezes'' some portion of a function's arguments and/or keywords
resulting in a new object with a simplified signature.  For example,
\function{partial} can be used to create a callable that behaves like
the \function{int} function where the \var{base} argument defaults to
two:
  \begin{verbatim}
        >>> basetwo = partial(int, base=2)
        >>> basetwo.__doc__ = 'Convert base 2 string to an int.'
        >>> basetwo('10010')
        18
  \end{verbatim}
\end{funcdesc}

\begin{funcdesc}{update_wrapper}
{wrapper, wrapped\optional{, assigned}\optional{, updated}}
Update a \var{wrapper} function to look like the \var{wrapped} function.
The optional arguments are tuples to specify which attributes of the original
function are assigned directly to the matching attributes on the wrapper
function and which attributes of the wrapper function are updated with
the corresponding attributes from the original function. The default
values for these arguments are the module level constants
\var{WRAPPER_ASSIGNMENTS} (which assigns to the wrapper function's
\var{__name__}, \var{__module__} and \var{__doc__}, the documentation string)
and \var{WRAPPER_UPDATES} (which updates the wrapper function's \var{__dict__},
i.e. the instance dictionary).

The main intended use for this function is in decorator functions
which wrap the decorated function and return the wrapper. If the
wrapper function is not updated, the metadata of the returned function
will reflect the wrapper definition rather than the original function
definition, which is typically less than helpful.
\end{funcdesc}

\begin{funcdesc}{wraps}
{wrapped\optional{, assigned}\optional{, updated}}
This is a convenience function for invoking
\code{partial(update_wrapper, wrapped=wrapped, assigned=assigned, updated=updated)}
as a function decorator when defining a wrapper function. For example:
  \begin{verbatim}
        >>> def my_decorator(f):
        ...     @wraps(f)
        ...     def wrapper(*args, **kwds):
        ...         print 'Calling decorated function'
        ...         return f(*args, **kwds)
        ...     return wrapper
        ...
        >>> @my_decorator
        ... def example():
	...     """Docstring"""
        ...     print 'Called example function'
        ...
        >>> example()
        Calling decorated function
        Called example function
        >>> example.__name__
        'example'
	>>> example.__doc__
	'Docstring'
  \end{verbatim}
Without the use of this decorator factory, the name of the example
function would have been \code{'wrapper'}, and the docstring of the
original \function{example()} would have been lost.
\end{funcdesc}


\subsection{\class{partial} Objects \label{partial-objects}}


\class{partial} objects are callable objects created by \function{partial()}.
They have three read-only attributes:

\begin{memberdesc}[callable]{func}{}
A callable object or function.  Calls to the \class{partial} object will
be forwarded to \member{func} with new arguments and keywords.
\end{memberdesc}

\begin{memberdesc}[tuple]{args}{}
The leftmost positional arguments that will be prepended to the
positional arguments provided to a \class{partial} object call.
\end{memberdesc}

\begin{memberdesc}[dict]{keywords}{}
The keyword arguments that will be supplied when the \class{partial} object
is called.
\end{memberdesc}

\class{partial} objects are like \class{function} objects in that they are
callable, weak referencable, and can have attributes.  There are some
important differences.  For instance, the \member{__name__} and
\member{__doc__} attributes are not created automatically.  Also,
\class{partial} objects defined in classes behave like static methods and
do not transform into bound methods during instance attribute look-up.
