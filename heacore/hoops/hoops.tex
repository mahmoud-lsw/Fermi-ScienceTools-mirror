\documentstyle[11pt]{book}
%
% $Id: html.sty,v 1.1.1.1 2012/02/21 18:10:24 elwinter Exp $
% LaTeX2HTML Version 96.2 : html.sty
% 
% This file contains definitions of LaTeX commands which are
% processed in a special way by the translator. 
% For example, there are commands for embedding external hypertext links,
% for cross-references between documents or for including raw HTML.
% This file includes the comments.sty file v2.0 by Victor Eijkhout
% In most cases these commands do nothing when processed by LaTeX.
%
% Place this file in a directory accessible to LaTeX (i.e., somewhere
% in the TEXINPUTS path.)
%
% NOTE: This file works with LaTeX 2.09 or (the newer) LaTeX2e.
%       If you only have LaTeX 2.09, some complex LaTeX2HTML features
%       like support for segmented documents are not available.

% Changes:
% See the change log at end of file.


% Exit if the style file is already loaded
% (suggested by Lee Shombert <las@potomac.wash.inmet.com>
\ifx \htmlstyloaded\relax \endinput\else\let\htmlstyloaded\relax\fi
\makeatletter

\newcommand{\latextohtml}{\LaTeX2\texttt{HTML}}


%%% LINKS TO EXTERNAL DOCUMENTS
%
% This can be used to provide links to arbitrary documents.
% The first argumment should be the text that is going to be
% highlighted and the second argument a URL.
% The hyperlink will appear as a hyperlink in the HTML 
% document and as a footnote in the dvi or ps files.
%
\newcommand{\htmladdnormallinkfoot}[2]{#1\footnote{#2}} 


% This is an alternative definition of the command above which
% will ignore the URL in the dvi or ps files.
\newcommand{\htmladdnormallink}[2]{#1}


% This command takes as argument a URL pointing to an image.
% The image will be embedded in the HTML document but will
% be ignored in the dvi and ps files.
%
\newcommand{\htmladdimg}[1]{}


%%% CROSS-REFERENCES BETWEEN (LOCAL OR REMOTE) DOCUMENTS
%
% This can be used to refer to symbolic labels in other Latex 
% documents that have already been processed by the translator.
% The arguments should be:
% #1 : the URL to the directory containing the external document
% #2 : the path to the labels.pl file of the external document.
% If the external document lives on a remote machine then labels.pl 
% must be copied on the local machine.
%
%e.g. \externallabels{http://cbl.leeds.ac.uk/nikos/WWW/doc/tex2html/latex2html}
%                    {/usr/cblelca/nikos/tmp/labels.pl}
% The arguments are ignored in the dvi and ps files.
%
\newcommand{\externallabels}[2]{}


% This complements the \externallabels command above. The argument
% should be a label defined in another latex document and will be
% ignored in the dvi and ps files.
%
\newcommand{\externalref}[1]{}


% Suggested by  Uffe Engberg (http://www.brics.dk/~engberg/)
% This allows the same effect for citations in external bibliographies.
% An  \externallabels  command must be given, locating a labels.pl file
% which defines the location and keys used in the external .html file.
%  
\newcommand{\externalcite}{\nocite}


%%% HTMLRULE
% This command adds a horizontal rule and is valid even within
% a figure caption.
% Here we introduce a stub for compatibility.
\newcommand{\htmlrule}{\protect\HTMLrule}
\newcommand{\HTMLrule}{\@ifstar\@htmlrule\@htmlrule}
\newcommand{\@htmlrule}[1]{}

% This command adds information within the <BODY> ... </BODY> tag
%
\newcommand{\bodytext}[1]{}
\newcommand{\htmlbody}{}


%%% HYPERREF 
% Suggested by Eric M. Carol <eric@ca.utoronto.utcc.enfm>
% Similar to \ref but accepts conditional text. 
% The first argument is HTML text which will become ``hyperized''
% (underlined).
% The second and third arguments are text which will appear only in the paper
% version (DVI file), enclosing the fourth argument which is a reference to a label.
%
%e.g. \hyperref{using the tracer}{using the tracer (see Section}{)}{trace}
% where there is a corresponding \label{trace}
%
\newcommand{\hyperref}{\hyperrefx[ref]}
\def\hyperrefx[#1]{{\def\next{#1}%
 \def\tmp{ref}\ifx\next\tmp\aftergroup\hyperrefref
 \else\def\tmp{pageref}\ifx\next\tmp\aftergroup\hyperpageref
 \else\def\tmp{page}\ifx\next\tmp\aftergroup\hyperpageref
 \else\def\tmp{noref}\ifx\next\tmp\aftergroup\hypernoref
 \else\def\tmp{no}\ifx\next\tmp\aftergroup\hypernoref
 \else\typeout{*** unknown option \next\space to  hyperref ***}%
 \fi\fi\fi\fi\fi}}
\newcommand{\hyperrefref}[4]{#2\ref{#4}#3}
\newcommand{\hyperpageref}[4]{#2\pageref{#4}#3}
\newcommand{\hypernoref}[3]{#2}


%%% HYPERCITE --- added by RRM
% Suggested by Stephen Simpson <simpson@math.psu.edu>
% effects the same ideas as in  \hyperref, but for citations.
% It does not allow an optional argument to the \cite, in LaTeX.
%
%   \hypercite{<html-text>}{<LaTeX-text>}{<opt-text>}{<key>}
%
% uses the pre/post-texts in LaTeX, with a  \cite{<key>}
%
%   \hypercite[ext]{<html-text>}{<LaTeX-text>}{<key>}
%
% uses the pre/post-texts in LaTeX, with a  \nocite{<key>}
% the actual reference comes from an \externallabels  file.
%
\newcommand{\hypercite}{\hypercitex[int]}
\def\hypercitex[#1]{{\def\next{#1}%
 \def\tmp{int}\ifx\next\tmp\aftergroup\hyperciteint
 \else\def\tmp{cite}\ifx\next\tmp\aftergroup\hyperciteint
 \else\def\tmp{ext}\ifx\next\tmp\aftergroup\hyperciteext
 \else\def\tmp{nocite}\ifx\next\tmp\aftergroup\hyperciteext
 \else\def\tmp{no}\ifx\next\tmp\aftergroup\hyperciteext
 \else\typeout{*** unknown option \next\space to  hypercite ***}%
 \fi\fi\fi\fi\fi}}
\newcommand{\hyperciteint}[4]{#2{\def\tmp{#3}\def\emptyopt{}%
 \ifx\tmp\emptyopt\cite{#4}\else\cite[#3]{#4}\fi}}
\newcommand{\hyperciteext}[3]{#2\nocite{#3}}



%%% HTMLREF
% Reference in HTML version only.
% Mix between \htmladdnormallink and \hyperref.
% First arg is text for in both versions, second is label for use in HTML
% version.
\newcommand{\htmlref}[2]{#1}

%%% HTMLCITE
% Reference in HTML version only.
% Mix between \htmladdnormallink and \hypercite.
% First arg is text for in both versions, second is citation for use in HTML
% version.
\newcommand{\htmlcite}[2]{#1}


%%% HTMLIMAGE
% This command can be used inside any environment that is converted
% into an inlined image (eg a "figure" environment) in order to change
% the way the image will be translated. The argument of \htmlimage
% is really a string of options separated by commas ie 
% [scale=<scale factor>],[external],[thumbnail=<reduction factor>
% The scale option allows control over the size of the final image.
% The ``external'' option will cause the image not to be inlined 
% (images are inlined by default). External images will be accessible
% via a hypertext link. 
% The ``thumbnail'' option will cause a small inlined image to be 
% placed in the caption. The size of the thumbnail depends on the
% reduction factor. The use of the ``thumbnail'' option implies
% the ``external'' option.
%
% Example:
% \htmlimage{scale=1.5,external,thumbnail=0.2}
% will cause a small thumbnail image 1/5th of the original size to be
% placed in the final document, pointing to an external image 1.5
% times bigger than the original.
% 
\newcommand{\htmlimage}[1]{}


% \htmlborder causes a border to be placed around an image or table
% when the image is placed within a <TABLE> cell.
\newcommand{\htmlborder}[1]{}

% Put \begin{makeimage}, \end{makeimage} around LaTeX to ensure its
% translation into an image.
% This shields sensitive text from being translated.
\newenvironment{makeimage}{}{}


%%% HTMLADDTONAVIGATION
% This command appends its argument to the buttons in the navigation
% panel. It is ignored by LaTeX.
%
% Example:
% \htmladdtonavigation{\htmladdnormallink
%              {\htmladdimg{http://server/path/to/gif}}
%              {http://server/path}}
\newcommand{\htmladdtonavigation}[1]{}


%%%%%%%%%%%%%%%%%%%%%%%%%%%%%%%%%%%%%%%%%%%%%%%%%%%%%%%%%%%%%%%%%%
% Comment.sty   version 2.0, 19 June 1992
% selectively in/exclude pieces of text: the user can define new
% comment versions, and each is controlled separately.
% This style can be used with plain TeX or LaTeX, and probably
% most other packages too.
%
% Examples of use in LaTeX and TeX follow \endinput
%
% Author
%    Victor Eijkhout
%    Department of Computer Science
%    University Tennessee at Knoxville
%    104 Ayres Hall
%    Knoxville, TN 37996
%    USA
%
%    eijkhout@cs.utk.edu
%
% Usage: all text included in between
%    \comment ... \endcomment
% or \begin{comment} ... \end{comment}
% is discarded. The closing command should appear on a line
% of its own. No starting spaces, nothing after it.
% This environment should work with arbitrary amounts
% of comment.
%
% Other 'comment' environments are defined by
% and are selected/deselected with
% \includecomment{versiona}
% \excludecoment{versionb}
%
% These environments are used as
% \versiona ... \endversiona
% or \begin{versiona} ... \end{versiona}
% with the closing command again on a line of its own.
%
% Basic approach:
% to comment something out, scoop up  every line in verbatim mode
% as macro argument, then throw it away.
% For inclusions, both the opening and closing comands
% are defined as noop
%
% Changed \next to \html@next to prevent clashes with other sty files
% (mike@emn.fr)
% Changed \html@next to \htmlnext so the \makeatletter and
% \makeatother commands could be removed (they were causing other
% style files - changebar.sty - to crash) (nikos@cbl.leeds.ac.uk)
% Changed \htmlnext back to \html@next...

\def\makeinnocent#1{\catcode`#1=12 }
\def\csarg#1#2{\expandafter#1\csname#2\endcsname}

\def\ThrowAwayComment#1{\begingroup
    \def\CurrentComment{#1}%
    \let\do\makeinnocent \dospecials
    \makeinnocent\^^L% and whatever other special cases
    \endlinechar`\^^M \catcode`\^^M=12 \xComment}
{\catcode`\^^M=12 \endlinechar=-1 %
 \gdef\xComment#1^^M{\def\test{#1}\edef\test{\meaning\test}
      \csarg\ifx{PlainEnd\CurrentComment Test}\test
          \let\html@next\endgroup
      \else \csarg\ifx{LaLaEnd\CurrentComment Test}\test
            \edef\html@next{\endgroup\noexpand\end{\CurrentComment}}
      \else \csarg\ifx{LaInnEnd\CurrentComment Test}\test
            \edef\html@next{\endgroup\noexpand\end{\CurrentComment}}
      \else \let\html@next\xComment
      \fi \fi \fi \html@next}
}

\def\includecomment
 #1{\expandafter\def\csname#1\endcsname{}%
    \expandafter\def\csname end#1\endcsname{}}
\def\excludecomment
 #1{\expandafter\def\csname#1\endcsname{\ThrowAwayComment{#1}}%
    {\escapechar=-1\relax
     \edef\tmp{\string\\end#1}%
      \csarg\xdef{PlainEnd#1Test}{\meaning\tmp}%
     \edef\tmp{\string\\end\string\{#1\string\}}%
      \csarg\xdef{LaLaEnd#1Test}{\meaning\tmp}%
     \edef\tmp{\string\\end \string\{#1\string\}}%
      \csarg\xdef{LaInnEnd#1Test}{\meaning\tmp}%
    }}

\excludecomment{comment}
%%%%%%%%%%%%%%%%%%%%%%%%%%%%%%%%%%%%%%%%%%%%%%%%%%%%%%%%%%%%%%%%%%%%%%%%%%%
% end Comment.sty
%%%%%%%%%%%%%%%%%%%%%%%%%%%%%%%%%%%%%%%%%%%%%%%%%%%%%%%%%%%%%%%%%%%%%%%%%%%


%%% RAW HTML 
% 
% Enclose raw HTML between a \begin{rawhtml} and \end{rawhtml}.
% The html environment ignores its body
%
\excludecomment{rawhtml}


%%% HTML ONLY
%
% Enclose LaTeX constructs which will only appear in the 
% HTML output and will be ignored by LaTeX with 
% \begin{htmlonly} and \end{htmlonly}
%
\excludecomment{htmlonly}
% Shorter version
\newcommand{\html}[1]{}

% for images.tex only
\excludecomment{imagesonly}

%%% LaTeX ONLY
% Enclose LaTeX constructs which will only appear in the 
% DVI output and will be ignored by latex2html with 
%\begin{latexonly} and \end{latexonly}
%
\newenvironment{latexonly}{}{}
% Shorter version
\newcommand{\latex}[1]{#1}


%%% LaTeX or HTML
% Combination of \latex and \html.
% Say \latexhtml{this should be latex text}{this html text}
%
%\newcommand{\latexhtml}[2]{#1}
\long\def\latexhtml#1#2{#1}


%%% tracing the HTML conversions
% This alters the tracing-level within the processing
% performed by  latex2html  by adjusting  $VERBOSITY
% (see  latex2html.config  for the appropriate values)
%
\newcommand{\htmltracing}[1]{}
\newcommand{\htmltracenv}[1]{}


%%%  \strikeout for HTML only
% uses <STRIKE>...</STRIKE> tags on the argument
% LaTeX just gobbles it up.
\newcommand{\strikeout}[1]{}


%%%%%%%%%%%%%%%%%%%%%%%%%%%%%%%%%%%%%%%%%%%%%%%%%%%%%%%%%%%%%%%%%%
%%% JCL - stop input here if LaTeX2e is not present
%%%%%%%%%%%%%%%%%%%%%%%%%%%%%%%%%%%%%%%%%%%%%%%%%%%%%%%%%%%%%%%%%%
\ifx\if@compatibility\undefined
  %LaTeX209
  \makeatother\relax\expandafter\endinput
\fi
\if@compatibility
  %LaTeX2e in LaTeX209 compatibility mode
  \makeatother\relax\expandafter\endinput
\fi

%%%%%%%%%%%%%%%%%%%%%%%%%%%%%%%%%%%%%%%%%%%%%%%%%%%%%%%%%%%%%%%%%%
%
% Start providing LaTeX2e extension:
% This is currently:
%  - additional optional argument for \htmladdimg
%  - support for segmented documents
%

\ProvidesPackage{html}
          [1996/12/22 v1.1 hypertext commands for latex2html (nd, hws, rrm)]
%%%%MG

% This command takes as argument a URL pointing to an image.
% The image will be embedded in the HTML document but will
% be ignored in the dvi and ps files.  The optional argument
% denotes additional HTML tags.
%
% Example:  \htmladdimg[ALT="portrait" ALIGN=CENTER]{portrait.gif}
%
\renewcommand{\htmladdimg}[2][]{}

%%% HTMLRULE for LaTeX2e
% This command adds a horizontal rule and is valid even within
% a figure caption.
%
% This command is best used with LaTeX2e and HTML 3.2 support.
% It is like \hrule, but allows for options via key--value pairs
% as follows:  \htmlrule[key1=value1, key2=value2, ...] .
% Use \htmlrule* to suppress the <BR> tag.
% Eg. \htmlrule[left, 15, 5pt, "none", NOSHADE] produces
% <BR CLEAR="left"><HR NOSHADE SIZE="15">.
% Renew the necessary part.
\renewcommand{\@htmlrule}[1][all]{}

%%%%%%%%%%%%%%%%%%%%%%%%%%%%%%%%%%%%%%%%%%%%%%%%%%%%%%%%%%%%%%%%%%
%
%  renew some definitions to allow optional arguments
%
% The description of the options is missing, as yet.
%
\renewcommand{\latextohtml}{\textup{\LaTeX2\texttt{HTML}}}
\renewcommand{\htmladdnormallinkfoot}[3][]{#2\footnote{#3}} 
\renewcommand{\htmladdnormallink}[3][]{#2}
\renewcommand{\htmlbody}[1][]{}
\renewcommand{\hyperref}[1][ref]{\hyperrefx[#1]}
\renewcommand{\hypercite}[1][int]{\hypercitex[#1]}
\renewcommand{\htmlref}[3][]{#2}
\renewcommand{\htmlcite}[1]{#1\htmlcitex}
\newcommand{\htmlcitex}[2][]{{\def\tmp{#1}\ifx\tmp\@empty\else~[#1]\fi}}
\renewcommand{\htmlimage}[2][]{}
\renewcommand{\htmlborder}[2][]{}


%%%%%%%%%%%%%%%%%%%%%%%%%%%%%%%%%%%%%%%%%%%%%%%%%%%%%%%%%%%%%%%%%%
%
%  HTML  HTMLset  HTMLsetenv
%
%  These commands do nothing in LaTeX, but can be used to place
%  HTML tags or set Perl variables during the LaTeX2HTML processing;
%  They are intended for expert use only.

\newcommand{\HTML}[2][]{}
\newcommand{\HTMLset}[3][]{}
\newcommand{\HTMLsetenv}[3][]{}

%%%%%%%%%%%%%%%%%%%%%%%%%%%%%%%%%%%%%%%%%%%%%%%%%%%%%%%%%%%%%%%%%%
%
% The following commands pertain to document segmentation, and
% were added by Herbert Swan <dprhws@edp.Arco.com> (with help from
% Michel Goossens <goossens@cern.ch>):
%
%
% This command inputs internal latex2html tables so that large
% documents can to partitioned into smaller (more manageable)
% segments.
%
\newcommand{\internal}[2][internals]{}

%
%  Define a dummy stub \htmlhead{}.  This command causes latex2html
%  to define the title of the start of a new segment.  It is not
%  normally placed in the user's document.  Rather, it is passed to
%  latex2html via a .ptr file written by \segment.
%
\newcommand{\htmlhead}[3][]{}

%  In the LaTeX2HTML version this will eliminate the title line
%  generated by a \segment command, but retains the title string
%  for use in other places.
%
\newcommand{\htmlnohead}{}


%  In the LaTeX2HTML version this put a URL into a <BASE> tag
%  within the <HEAD>...</HEAD> portion of a document.
%
\newcommand{\htmlbase}[1]{}
%

%
%  The dummy command \endpreamble is needed by latex2html to
%  mark the end of the preamble in document segments that do
%  not contain a \begin{document}
%
\newcommand{\startdocument}{}

%
%  These do nothing in LaTeX but for LaTeX2HTML
%  they mark where the table of child-links and info-page 
%  should be placed, when the user wants other than the default.
%  The *-versions omit the preceding <BR> tag.
%
\newcommand{\tableofchildlinks}{\@ifstar\@tochildlinks\@tochildlinks}
\newcommand{\@tochildlinks}{}

\newcommand{\htmlinfo}{\@ifstar\@htmlinfo\@htmlinfo}
\newcommand{\@htmlinfo}{}

%
%  Allocate a new set of section counters, which will get incremented
%  for "*" forms of sectioning commands, and for a few miscellaneous
%  commands.
%

\newcounter{lpart}
\newcounter{lchapter}[part]
\@ifundefined{chapter}{\newcounter{lsection}[part]}{\newcounter{lsection}[chapter]}
\newcounter{lsubsection}[section]
\newcounter{lsubsubsection}[subsection]
\newcounter{lparagraph}[subsubsection]
\newcounter{lsubparagraph}[paragraph]
\newcounter{lequation}

%
%  Redefine "*" forms of sectioning commands to increment their
%  respective counters.
%
\let\Hpart=\part
\let\Hchapter=\chapter
\let\Hsection=\section
\let\Hsubsection=\subsection
\let\Hsubsubsection=\subsubsection
\let\Hparagraph=\paragraph
\let\Hsubparagraph=\subparagraph
\let\Hsubsubparagraph=\subsubparagraph

\ifx\c@subparagraph\undefined
 \newcounter{lsubsubparagraph}[lsubparagraph]
\else
 \newcounter{lsubsubparagraph}[subparagraph]
\fi

%
%  The following definitions are specific to LaTeX2e:
%  (They must be commented out for LaTeX 2.09)
%
\renewcommand{\part}{\@ifstar{\stepcounter{lpart}%
  \bgroup\def\tmp{*}\H@part}{\bgroup\def\tmp{}\H@part}}
\newcommand{\H@part}[1][]{%
 \expandafter\egroup\expandafter\Hpart\tmp}

\ifx\Hchapter\relax\else\@ifundefined{chapter}{}{%
 \def\chapter{\resetsections \@ifstar{\stepcounter{lchapter}%
   \bgroup\def\tmp{*}\H@chapter}{\bgroup\def\tmp{}\H@chapter}}}\fi
\newcommand{\H@chapter}[1][]{%
 \expandafter\egroup\expandafter\Hchapter\tmp}

\renewcommand{\section}{\resetsubsections
 \@ifstar{\stepcounter{lsection}\bgroup\def\tmp{*}%
   \H@section}{\bgroup\def\tmp{}\H@section}}
\newcommand{\H@section}[1][]{%
 \expandafter\egroup\expandafter\Hsection\tmp}

\renewcommand{\subsection}{\resetsubsubsections
 \@ifstar{\stepcounter{lsubsection}\bgroup\def\tmp{*}%
   \H@subsection}{\bgroup\def\tmp{}\H@subsection}}
\newcommand{\H@subsection}[1][]{%
 \expandafter\egroup\expandafter\Hsubsection\tmp}

\renewcommand{\subsubsection}{\resetparagraphs
 \@ifstar{\stepcounter{lsubsubsection}\bgroup\def\tmp{*}%
   \H@subsubsection}{\bgroup\def\tmp{}\H@subsubsection}}
\newcommand{\H@subsubsection}[1][]{%
 \expandafter\egroup\expandafter\Hsubsubsection\tmp}

\renewcommand{\paragraph}{\resetsubparagraphs
 \@ifstar{\stepcounter{lparagraph}\bgroup\def\tmp{*}%
   \H@paragraph}{\bgroup\def\tmp{}\H@paragraph}}
\newcommand\H@paragraph[1][]{%
 \expandafter\egroup\expandafter\Hparagraph\tmp}

\renewcommand{\subparagraph}{\resetsubsubparagraphs
 \@ifstar{\stepcounter{lsubparagraph}\bgroup\def\tmp{*}%
   \H@subparagraph}{\bgroup\def\tmp{}\H@subparagraph}}
\newcommand\H@subparagraph[1][]{%
 \expandafter\egroup\expandafter\Hsubparagraph\tmp}

\ifx\Hsubsubparagraph\relax\else\@ifundefined{subsubparagraph}{}{%
\def\subsubparagraph{%
 \@ifstar{\stepcounter{lsubsubparagraph}\bgroup\def\tmp{*}%
   \H@subsubparagraph}{\bgroup\def\tmp{}\H@subsubparagraph}}}\fi
\newcommand\H@subsubparagraph[1][]{%
 \expandafter\egroup\expandafter\Hsubsubparagraph\tmp}
%
\def\resetsections{\setcounter{section}{0}\setcounter{lsection}{0}%
 \reset@dependents{section}\resetsubsections }
\def\resetsubsections{\setcounter{subsection}{0}\setcounter{lsubsection}{0}%
 \reset@dependents{subsection}\resetsubsubsections }
\def\resetsubsubsections{\setcounter{subsubsection}{0}\setcounter{lsubsubsection}{0}%
 \reset@dependents{subsubsection}\resetparagraphs }
%
\def\resetparagraphs{\setcounter{lparagraph}{0}\setcounter{lparagraph}{0}%
 \reset@dependents{paragraph}\resetsubparagraphs }
\def\resetsubparagraphs{\ifx\c@subparagraph\undefined\else
  \setcounter{subparagraph}{0}\fi \setcounter{lsubparagraph}{0}%
 \reset@dependents{subparagraph}\resetsubsubparagraphs }
\def\resetsubsubparagraphs{\ifx\c@subsubparagraph\undefined\else
  \setcounter{subsubparagraph}{0}\fi \setcounter{lsubsubparagraph}{0}}
%
\def\reset@dependents#1{\begingroup\let \@elt \@stpelt
 \csname cl@#1\endcsname\endgroup}
%
%
%  Define a helper macro to dump a single \secounter command to a file.
%
\newcommand{\DumpPtr}[2]{%
\count255=\arabic{#1}\def\dummy{dummy}\def\tmp{#2}%
\ifx\tmp\dummy\else\advance\count255 by \arabic{#2}\fi
\immediate\write\ptrfile{%
\noexpand\setcounter{#1}{\number\count255}}}

%
%  Define a helper macro to dump all counters to the file.
%  The value for each counter will be the sum of the l-counter
%      actual LaTeX section counter.
%  Also dump an \htmlhead{section-command}{section title} command
%      to the file.
%
\newwrite\ptrfile
\def\DumpCounters#1#2#3#4{%
\begingroup\let\protect=\noexpand
\immediate\openout\ptrfile = #1.ptr
\DumpPtr{part}{lpart}%
\ifx\Hchapter\relax\else\DumpPtr{chapter}{lchapter}\fi
\DumpPtr{section}{lsection}%
\DumpPtr{subsection}{lsubsection}%
\DumpPtr{subsubsection}{lsubsubsection}%
\DumpPtr{paragraph}{lparagraph}%
\DumpPtr{subparagraph}{lsubparagraph}%
\DumpPtr{equation}{lequation}%
\DumpPtr{footnote}{dummy}%
\def\tmp{#4}\ifx\tmp\@empty
\immediate\write\ptrfile{\noexpand\htmlhead{#2}{#3}}\else
\immediate\write\ptrfile{\noexpand\htmlhead[#4]{#2}{#3}}\fi
\dumpcitestatus \dumpcurrentcolor
\immediate\closeout\ptrfile
\endgroup }


%% interface to natbib.sty

\def\dumpcitestatus{}
\def\loadcitestatus{\def\dumpcitestatus{%
  \ifciteindex\immediate\write\ptrfile{\noexpand\citeindextrue}%
  \else\immediate\write\ptrfile{\noexpand\citeindexfalse}\fi }%
}
\@ifpackageloaded{natbib}{\loadcitestatus}{%
 \AtBeginDocument{\@ifpackageloaded{natbib}{\loadcitestatus}{}}}


%% interface to color.sty

\def\dumpcurrentcolor{}
\def\loadsegmentcolors{%
 \let\real@pagecolor=\pagecolor
 \let\pagecolor\segmentpagecolor
 \let\segmentcolor\color
 \ifx\current@page@color\undefined \def\current@page@color{{}}\fi
 \def\dumpcurrentcolor{\bgroup\def\@empty@{{}}%
   \expandafter\def\expandafter\tmp\space####1@{\def\thiscol{####1}}%
  \ifx\current@color\@empty@\def\thiscol{}\else
   \expandafter\tmp\current@color @\fi
  \immediate\write\ptrfile{\noexpand\segmentcolor{\thiscol}}%
  \ifx\current@page@color\@empty@\def\thiscol{}\else
   \expandafter\tmp\current@page@color @\fi
  \immediate\write\ptrfile{\noexpand\segmentpagecolor{\thiscol}}%
 \egroup}%
 \global\let\loadsegmentcolors=\relax
}

% These macros are needed within  images.tex  since this inputs
% the <segment>.ptr files for a segment, so that counters are
% colors are synchronised.
%
\newcommand{\segmentpagecolor}[1][]{%
 \@ifpackageloaded{color}{\loadsegmentcolors\bgroup
  \def\tmp{#1}\ifx\@empty\tmp\def\next{[]}\else\def\next{[#1]}\fi
  \expandafter\segmentpagecolor@\next}%
 {\@gobble}}
\def\segmentpagecolor@[#1]#2{\def\tmp{#1}\def\tmpB{#2}%
 \ifx\tmpB\@empty\let\next=\egroup
 \else
  \let\realendgroup=\endgroup
  \def\endgroup{\edef\next{\noexpand\realendgroup
   \def\noexpand\current@page@color{\current@color}}\next}%
  \ifx\tmp\@empty\real@pagecolor{#2}\def\model{}%
  \else\real@pagecolor[#1]{#2}\def\model{[#1]}%
  \fi
  \edef\next{\egroup\def\noexpand\current@page@color{\current@page@color}%
  \noexpand\real@pagecolor\model{#2}}%
 \fi\next}
%
\newcommand{\segmentcolor}[2][named]{\@ifpackageloaded{color}%
 {\loadsegmentcolors\segmentcolor[#1]{#2}}{}}

\@ifpackageloaded{color}{\loadsegmentcolors}{\let\real@pagecolor=\@gobble
 \AtBeginDocument{\@ifpackageloaded{color}{\loadsegmentcolors}{}}}


%  Define the \segment[align]{file}{section-command}{section-title} command,
%  and its helper macros.  This command does four things:
%       1)  Begins a new LaTeX section;
%       2)  Writes a list of section counters to file.ptr, each
%           of which represents the sum of the LaTeX section
%           counters, and the l-counters, defined above;
%       3)  Write an \htmlhead{section-title} command to file.ptr;
%       4)  Inputs file.tex.


\def\segment{\@ifstar{\@@htmls}{\@@html}}
\newcommand{\@@htmls}[1][]{\@@htmlsx{#1}}
\newcommand{\@@html}[1][]{\@@htmlx{#1}}
\def\@@htmlsx#1#2#3#4{\csname #3\endcsname* {#4}%
                   \DumpCounters{#2}{#3*}{#4}{#1}\input{#2}}
\def\@@htmlx#1#2#3#4{\csname #3\endcsname {#4}%
                   \DumpCounters{#2}{#3}{#4}{#1}\input{#2}}

\makeatother
\endinput


% Modifications:
%
% (The listing of Initiales see Changes)

% $Log: html.sty,v $
% Revision 1.1.1.1  2012/02/21 18:10:24  elwinter
% Import of CLHEP 2.1.0.1.
%
% Revision 1.2  2010/06/16 17:12:28  garren
% merging changes from 2.0.5.0.b01
%
% Revision 1.1.6.2  2009/11/13 20:29:42  garren
% get the changes from 2.0.4.4
%
% Revision 1.1.2.1  2009/11/12 19:52:56  garren
% make pdf documents
%
% Revision 1.2  1997/07/10 06:16:42  RRM
%      LaTeX styles, used in the v97.1 manual
%
% Revision 1.17  1997/07/08 11:23:39  RRM
%     include value of footnote counter in .ptr files for segments
%
% Revision 1.16  1997/07/03 08:56:34  RRM
%     use \textup  within the \latextohtml macro
%
% Revision 1.15  1997/06/15 10:24:58  RRM
%      new command  \htmltracenv  as environment-ordered \htmltracing
%
% Revision 1.14  1997/06/06 10:30:37  RRM
%  -   new command:  \htmlborder  puts environment into a <TABLE> cell
%      with a border of specified width, + other attributes.
%  -   new commands: \HTML  for setting arbitrary HTML tags, with attributes
%                    \HTMLset  for setting Perl variables, while processing
%                    \HTMLsetenv  same as \HTMLset , but it gets processed
%                                 as if it were an environment.
%  -   new command:  \latextohtml  --- to set the LaTeX2HTML name/logo
%  -   fixed some remaining problems with \segmentcolor & \segmentpagecolor
%
% Revision 1.13  1997/05/19 13:55:46  RRM
%      alterations and extra options to  \hypercite
%
% Revision 1.12  1997/05/09 12:28:39  RRM
%  -  Added the optional argument to \htmlhead, also in \DumpCounters
%  -  Implemented \HTMLset as a no-op in LaTeX.
%  -  Fixed a bug in accessing the page@color settings.
%
% Revision 1.11  1997/03/26 09:32:40  RRM
%  -  Implements LaTeX versions of  \externalcite  and  \hypercite  commands.
%     Thanks to  Uffe Engberg  and  Stephen Simpson  for the suggestions.
%
% Revision 1.10  1997/03/06 07:37:58  RRM
% Added the  \htmltracing  command, for altering  $VERBOSITY .
%
% Revision 1.9  1997/02/17 02:26:26  RRM
% - changes to counter handling (RRM)
% - shuffled around some definitions
% - changed \htmlrule of 209 mode
%
% Revision 1.8  1997/01/26 09:04:12  RRM
% RRM: added optional argument to sectioning commands
%      \htmlbase  sets the <BASE HREF=...> tag
%      \htmlinfo  and  \htmlinfo* allow the document info to be positioned
%
% Revision 1.7  1997/01/03 12:15:44  L2HADMIN
% % - fixes to the  color  and  natbib  interfaces
% % - extended usage of  \hyperref, via an optional argument.
% % - extended use comment environments to allow shifting expansions
% %     e.g. within \multicolumn  (`bug' reported by Luc De Coninck).
% % - allow optional argument to: \htmlimage, \htmlhead,
% %     \htmladdimg, \htmladdnormallink, \htmladdnormallinkfoot
% % - added new commands: \htmlbody, \htmlnohead
% % - added new command: \tableofchildlinks
%
% Revision 1.6  1996/12/25 03:04:54  JCL
% added patches to segment feature from Martin Wilck
%
% Revision 1.5  1996/12/23 01:48:06  JCL
%  o introduced the environment makeimage, which may be used to force
%    LaTeX2HTML to generate an image from the contents.
%    There's no magic, all what we have now is a defined empty environment
%    which LaTeX2HTML will not recognize and thus pass it to images.tex.
%  o provided \protect to the \htmlrule commands to allow for usage
%    within captions.
%
% Revision 1.4  1996/12/21 19:59:22  JCL
% - shuffled some entries
% - added \latexhtml command
%
% Revision 1.3  1996/12/21 12:22:59  JCL
% removed duplicate \htmlrule, changed \htmlrule back not to create a \hrule
% to allow occurrence in caption
%
% Revision 1.2  1996/12/20 04:03:41  JCL
% changed occurrence of \makeatletter, \makeatother
% added new \htmlrule command both for the LaTeX2.09 and LaTeX2e
% sections
%
%
% jcl 30-SEP-96
%  - Stuck the commands commonly used by both LaTeX versions to the top,
%    added a check which stops input or reads further if the document
%    makes use of LaTeX2e.
%  - Introduced rrm's \dumpcurrentcolor and \bodytext
% hws 31-JAN-96 - Added support for document segmentation
% hws 10-OCT-95 - Added \htmlrule command
% jz 22-APR-94 - Added support for htmlref
% nd  - Created

\htmladdtonavigation
   {\begin{rawhtml}
 <A HREF="http://heasarc.gsfc.nasa.gov/docs/software/lheasoft">HEAsoft Home</A>
    \end{rawhtml}}
\oddsidemargin=0.00in
\evensidemargin=0.00in
\textwidth=6.5in
\topmargin=0.0in
\textheight=8.5in
\parindent=0cm
\parskip=0.2cm
\begin{document}

\begin{titlepage}
\normalsize
\vspace*{4.0cm}
\begin{center}
{\Huge \bf HOOPS Developers Guide}\\
\end{center}
\medskip 
\medskip
\begin{center}
{\Large Version 0.9 \\}
\end{center}
\bigskip
\vskip 2.5cm
\begin{center}
{HEASARC\\
Code 662\\
Goddard Space Flight Center\\
Greenbelt, MD 20771\\
USA}
\end{center}

\vfill
\bigskip
\begin{center}
{\Large April 2003\\}
\end{center}
\vfill
\end{titlepage}

\begin{titlepage}
\vspace*{7.6cm}
\vfill
\end{titlepage}

\pagenumbering{roman}

\tableofcontents

\pagenumbering{arabic}
\chapter{Overview of HOOPS Capabilities}

\section{Introduction}
HOOPS is an object-oriented library for the development
of user interfaces and high-level software control
structures. To the user it behaves similarly to most
IRAF-inspired parameter systems in use today in
astronomical software, such as HEASARC's Xanadu Parameter
Interface (XPI) and ISDC's Parameter Interface Library (PIL).
To the developer, HOOPS is intended
to provide convenient, flexible, extensible and powerful
access to IRAF-style parameter concepts, both in the
context of traditional parameter files and in a more
general way. In particular,
it is hoped that new missions will use HOOPS to put
the tried and true IRAF parameter paradigm to new and
novel uses.

\subsection{HOOPS's relationship to ISDC's PIL}
HOOPS was originally conceived as an object-oriented
extension to PIL, to be used by GLAST mission-specific
software. Its current implementation uses PIL internally
for file access, command-line parsing, and user
interaction. However, it is intended eventually to
remove the dependency on PIL, for three reasons. First,
it is always easier to maintain a single software
component than to maintain two fairly tightly coupled
components. Second, PIL has (at the time of this writing)
some dependencies on Unix system libraries, and third,
the usage of PIL forces some unneccessary limitations
as well as some complexity and inefficiency into the
HOOPS implementation.

In anticipation of the eventual replacement of PIL by
native HOOPS methods, PIL usage is restricted to only two
classes, PILParFile and PILParPrompt. New classes which
use PIL a bit differently could be derived from these classes.
New interfaces for file access or prompting which do not
use PIL at all should instead be derived directly from the base
classes IParFile and IParPrompt, respectively.

\section{Summary of Class Families}
HOOPS is divided into five main class hierarchies. To
convey the overall design concept, the classes will be
presented in descending order from the highest level
to the lowest level of abstraction. The detailed
descriptions of each class may be read in any order,
however.

\begin{itemize}
\item ParPrompt: mechanisms for obtaining input from the user
\item ParFile: a group of parameters associated with a file
\item ParGroup: a collection of parameters
\item Par: encapsulation of a single parameter
\item GenBiDirItor: a universal bidirectional iterator
\item Prim: support classes to handle type conversions
\end{itemize}

All of these classes have virtual methods as necessary to
support polymorphic behavior, to facilitate extension
of class behavior through derivation. Only the declarations
for pure virtual classes and concrete factory classes are
considered public.

\subsection{The ParPrompt Family}
The abstract base class (ABC) of the ParPrompt family
is IParPrompt, which defines a standard interface for
user prompts. Presently included in the hierarchy
are two concrete derived classes, ParPrompt and
PILParPrompt. The first of these, ParPrompt, is
only partially implemented at present, but it will
eventually provide a complete implementation of the
IParPrompt interface. The second class, PILParPrompt,
is a complete implementation of the IParPrompt interface
which uses PIL. Due to some limitations in PIL, the
PILParPrompt interface can only function in the context
of a set of parameters associated with a parameter file.
When it is complete, the ParPrompt interface will be
more general, and allow prompting for any group of
parameters.

Also defined is an abstract factory, or "virtual constructor"
class, IParPromptFactory. At present its only concrete
subclass is PILParPromptFactory, which may be used to
create instances of PILParPrompt.

\subsection{The ParFile Family}
The ABC of the ParFile family is IParFile, which defines
a standard interface for parameter file access. Presently
included in the hierarchy are two concrete derived classes,
ParFile and PILParFile. The former is only partially
implemented at present, but it will eventually provide a
complete native implementation of the IParFile interface
which will not require PIL. The second class, PILParFile,
is a complete implementation of the IParFile interface which
uses PIL.

Also defined is an abstract factory, or "virtual constructor"
class, IParFileFactory. At present its only concrete
subclass is PILParFileFactory, which may be used to
create instances of PILParFile.

\subsection{The ParGroup Family}
The ABC of the ParGroup family is IParGroup, which
encapsulates the concept of a group of parameters, and
defines a standard interface for accessing parameters
contained within the group. Presently only one derived
class exists, ParGroup, which provides a simple but
complete implementation.

Both the IParPrompt and IParFile interfaces can produce
references to an IParGroup-derived object containing
the parameters for which they are prompting and providing
file access, respectively. This is so that software
which uses HOOPS can be easily separated into three
parts: drivers which prompt the user and determine what
course the software should take, file I/O sections, and
components which simply use the parameters. Most code
(i.e. general library code) will fall in the latter category,
and it can and should use IParGroup-derived objects only,
and leave the other tasks to higher-level, more specialized
components.

Also defined is an abstract factory, or "virtual constructor"
class, IParGroupFactory. At present no concrete factories
exist, so it is not possible to instantiate explicitly
a ParGroup object except internally to HOOPS.

\subsection{The Par Family}
The ABC for the Par family is IPar, which encapsulates
the concept of a single parameter, and defines a standard
interface for accessing that parameter. Presently only
one derived class exists, Par, which provides a complete
implementation using the Prim class family to perform
type conversions.

The IPar interface contains methods to access the
eight kinds of information found in an IRAF-compliant
parameter: Name, Type, Mode, Value, Minimum value, Maximum
value, Prompt string and Comment. In addition there are
overloaded methods to perform explicit conversions to and
from C++ primitive types and strings, as well as assignment
and conversion (cast) operators to perform such conversions
implicitly.

Also defined is an abstract factory, or "virtual constructor"
class, IParFactory. One concrete factory, ParFactory,
exists for the creation of Par objects.

\subsection{The BiDirItor Family}
The purpose of this hierarchy is to provide an
iterator type which may be used to iterate over
collections of parameters. In general, templated
iterator classes (from STL and elsewhere) have common
features, but do not have common base classes. Moreover
they are not generally polymorphic. The BiDirItor family
provides a means of wrapping any class which behaves like
an iterator inside a polymorphic class deriving from a
common base. This iterator may then be used to iterate
over a collection of any object type.

The BiDirItor family is a templated hierarchy, whose
ABC is IBiDirItor. This provides a standard interface
for a bidirectional iterator, with methods to provide
the standard operations ->, unary *, ++, --, ==, !=. This is
subclassed in the templated BiDirItor class, which
wraps a specific iterator type, as indicated in one of its
template arguments. Finally, GenBiDirItor, which also derives
from IBiDirItor, contains a pointer to a IBiDirItor.
Thus GenBiDirItor provides a single iterator which may
be used to iterate over any underlying iterator type.

\subsection{The Prim Family}
The Prim family currently consists of the classes
IPrim and Prim. The former is an ABC defining
an interface of overloaded abstract To() and From() methods
to convert the (unspecified) type represented by IPrim 
to and from C++ primitive types and strings.

The templated class Prim, derives from IPrim. It contains
a data member of the type given as its template argument, and
provides an implementation of the Prim interface for that type.
For example, Prim$<$int$>$ is a class which knows how to convert
its int member to and from other types, throwing exceptions
when such a conversion results or may result in a loss
of precision or accuracy of the converted value.

Because Prim defines conversions to and from ALL
C++ primitive types and strings, a fair number of
these conversions may result in a loss of precision or
accuracy of a converted value. For example, if a Prim$<$int$>$
object is converted to a floating point value, a loss of
precision will occur. In this situation, Prim$<$int$>$ will perform
the conversion, but then throw an exception. Similarly,
conversions which result in overflows, underflows, conversions
to a (potentially) smaller type, and conversions between signed
and unsigned integral types will all throw exceptions. Whenever
possible a sensible conversion will be performed before the
exception is thrown, allowing calling code easy recovery if
it wishes to ignore the error.

Although developed to meet the needs of the Par family,
the Prim family was developed separately in the hopes that
it may be reused completely outside the context of parameter
objects. Because it is a completely general templated class,
it may be used any time one wishes to convert between different
types with a measure of control over what types of conversions
are considered valid.

An abstract factory class IPrimFactory exists for IPrim
objects. At present there is one concrete factory, PrimFactory,
which creates Prim objects.

\section{The ParPrompt Family}
The abstract interface defined by the IParPrompt class
declares a number of methods to handle prompting. In
the standard IRAF modality, parameters may be set on
the command line, in which case the prompts are suppressed.
Hence this interface also contains methods for handling
command line argument arrays. Of course, the arguments
passed need not necessarily come literally from something
a user typed, but could be explicitly constructed strings.
Its public methods are:

\begin{itemize}
\item ~IParPrompt(); deletes all resources associated
with the prompt object, including the underlying parameter
group.
\item IParPrompt \& operator =(const IParPrompt \& p); deletes
all resources associated with the destination prompt object
before copying the source object.
\item IParPrompt \& Prompt(); prompt for all parameters
associated with this IParPrompt object. Returns the object.
\item IParPrompt \& Prompt(const std::string \& pname); prompt
only for the named parameter. Returns the object.
\item IParPrompt \& Prompt(const std::vector<std::string> \& pnames);
prompts for the collection of parameters named in the argument.
Returns the object.
\item int Argc() const; returns the number of command line
arguments known to this object.
\item char ** const Argv() const; returns a copy of the command
line arguments known to this object.
\item IParGroup \& Group(); returns the group of parameter
objects which is present in this prompt object. Note that
the reference which is returned is modifiable.
\item const IParGroup \& Group() const; returns the group of
parameter objects which is present in this prompt object. Note
that the reference which is returned is not modifiable.
\item IParPrompt \& SetArgc(int argc); set the number of
command line arguments contained in this prompt object.
\item IParPrompt \& SetArgv(char ** argv); set the command
line arguments contained in this prompt object. The values
of the array of strings passed in are copied.
\item IParGroup * SetGroup(IParGroup * group); set the
underlying group of parameters to the group given in
the argument. Returns the previous group. Note that when
the prompt object is deleted, it automatically deletes
its internal group. In other words, if SetGroup is used
to set the group used by the prompt object, that group is
henceforth "owned" by the prompt object.
\end{itemize}

In addition, there is an IParPromptFactory abstract
class with one concrete subclass, PILParPromptFactory.

\section{The ParFile Family}
The abstract interface defined by the IParFile class
declares a number of methods to write and read parameters
to and from parameter files. It is anticipated new file
types can be supported by subclassing IParFile. Its
public methods are:

\begin{itemize}
\item void Load(); clear the current group of parameters
and load the parameters from the currently selected parameter
file.
\item void Save() const; save the current group of parameters
to the currently selected parameter file.
\item const std::string \& Component() const; returns the name
of the current component, which is used to find the parameter
file.
\item IParGroup \& Group(); returns the underlying group of
parameters. Note that the reference returned is modifiable.
\item const IParGroup \& Group() const; returns the underlying
group of parameters. Note that the reference returned is
not modifiable.
\item IParFile \& SetComponent(const std::string \& comp); set
the name of the current component. This will be used to
determine the parameter file name.
\item IParGroup * SetGroup(IParGroup * group); set the
underlying group of parameters to the group given in
the argument. Returns the previous group. Note that when
the prompt object is deleted, it automatically deletes
its internal group. In other words, if SetGroup is used
to set the group used by the prompt object, that group is
henceforth "owned" by the prompt object.
\item GenParItor begin(); returns an iterator pointing to
the first (modifiable) parameter object in the group.
\item ConstGenParItor begin() const; returns an iterator
pointing to the first (non-modifiable) parameter object
in the group.
\item GenParItor end(); returns an iterator pointing to
one item past the last (modifiable) item in the group.
\item ConstGenParItor end() const; returns an iterator
pointing to one item past the last (non-modifiable) item
in the group.
\end{itemize}

In addition, there is an IParFileFactory abstract
class with one concrete subclass, PILParFileFactory.

\end{document}
